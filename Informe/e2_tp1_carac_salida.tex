\documentclass[e2_tp1_main.tex]{subfiles}

\begin{document}

\section{Caracter\'istica de salida}


Se realizaron mediciones de la tensi\'on y la corriente de salida para distintos valores de carga, obteni\'endose as\'i las curvas caracter\'isticas del circuito. Las mismas se realizaron para tres tensiones de regulaci\'on distintas: 9V (figura \ref{fig:cs9}), 12V (figura \ref{fig:cs12}) y 15V (figura \ref{fig:cs15}).

\begin{figure}[!htp]
	\centering
	\includegraphics[width=0.8\textwidth]
	{curvas_salida/e2_tp1_carac_salida_9V.png}
	\caption{Curva de salida calculada, simulada y medida, con $V_O|_{REG}=9$V}
	\label{fig:cs9}
\end{figure}

\begin{figure}[!htp]
	\centering
	\includegraphics[width=0.8\textwidth]
	{curvas_salida/e2_tp1_carac_salida_12V.png}
	\caption{Curva de salida calculada, simulada y medida, con $V_O|_{REG}=12$V}
	\label{fig:cs12}
\end{figure}


\begin{figure}[!htp]
	\centering
	\includegraphics[width=0.8\textwidth]
	{curvas_salida/e2_tp1_carac_salida_15V.png}
	\caption{Curva de salida calculada, simulada y medida, con $V_O|_{REG}=15$V}
	\label{fig:cs15}
\end{figure}


En las mediciones, se observa que la tensi\'on de corto circuito real fue menor a la simulada, que a su vez fue menor a la calculada. Recordando la expresi\'on de esta corriente:
\[ I_{O\, CC} = 	\left( \frac{V_{BE\, 4}}{R_4} \right) \cdot
				\left( 1 + \frac{R_2}{R_3} \right)\]

De aqu\'i resulta evidente que esta corriente depende considerablemente de la polarizaci\'on de $T_4$. Si bien en los c\'alculos se consider\'o $V_{BE}=0.7$V, en la simulaci\'on se observa que esta tensi\'on es de 0.65V, lo cual reduce el valor de $I_{O\,CC}$ un 7.14\%: de 1.2A a 1.11A, lo cual explica la diferencia entre el c\'alculo y la simulaci\'on. En cuanto a la diferencia entre la simulaci\'on y la medici\'on, puede atribuirse el error obtenido a la tolerancia de los componentes (sobre todo de $R_2$, cuya sensibilidad es particularmente alta, dado que $R_2$ es considerablemente menor a $R_3$).

Otra diferencia notable entre la curva calculada y las demas est\'a en la ca\'ida de tensi\'on que se observa en la salida incluso antes de entrar en foldback. Los cambios peque\~nos que se observan para corrientes menores a 1.5A se pueden atribuir a que la impedancia de salida no es exactamente 0, y por lo tanto la regulaci\'on de l\'inea no es del todo perfecta.  Pero para corrientes superiores, se comienza a observar un descenso mayor en la tensi\'on. Esto se debe mayormente a que tanto en la simulaci\'on como en las mediciones se utiliz\'o $V_I \simeq V_{O\, REG} + 3$V, pero para las corrientes m\'as grandes se observa que esta tensi\'on no polariza el circuito de la forma m\'as \'optima. En la simulaci\'on se verific\'o que al aumentar la tensi\'on de entrada a $V_I = V_O + 5V$, se obtiene una caracter\'istica mucho m\'as recta en regulaci\'on.
 

\end{document}