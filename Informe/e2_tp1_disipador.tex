\documentclass[e2_tp1_main.tex]{subfiles}

\begin{document}
\section{Cálculo del disipador}
\subsection{Introducción}

Cuando uno o varios componentes electrónicos de un circuito por cuestiones
de diseño deben disipar mucha potencia (traducida en calor al exterior),
generalmente dichos dispositivos no son capaces de hacerlo sin llegar
a la propia falla y/o romperse. Es por esto que para estos casos se
utilizan los llamados disipadores térmicos. Estos disipadores son
piezas generalmente metálicas que utilizando las leyes de la termodinámicas
son capaces de extraerle calor al componente y liberarlo al ambiente.
Existen disipadores de varias formas, tamaños y materiales.

\subsubsection{Fundamento matemático}

Gracias a la física, es posible establecer una analogía entre la ley
de Ohm para corrientes y la propagación térmica del calor en los distintos
materiales, de manera tal como podemos observar en la tabla siguiente:

\begin{table}[H]

\begin{centering}
\begin{tabular}{|c|c|}
\hline 
Ley de Ohm & Propagación térmica\tabularnewline
\hline 
\hline 
Intensidad de corriente (I) & Calor (W)\tabularnewline
\hline 
Tensión (V) & Temperatura (T)\tabularnewline
\hline 
Resistencia (R) & Resistencia Termica (R)\tabularnewline
\hline 
$V=I\cdot R$ & $T=W\cdot R$\tabularnewline
\hline 
\end{tabular}\caption{Analogía térmica-ley de Ohm.}
\par\end{centering}
\end{table}

Siendo el calor expresado en Watts, la temperatura expresada en ºC,
y la resistencia térmica expresada en ºC/W.

La resistencia térmica, al igual que la resistencia para la ley de
Ohm, es inherente a cada material en sí, de manera tal de que por
ejemplo un aislante térmico posee alta resistencia térmica, y un conductor
térmico perfecto posee baja resistencia térmica.

\subsection{Elección del disipador}

\subsubsection{Cálculo de Potencia máxima disipada }

Correspondiente al análisis que se hizo del circuito en puntos anteriores,
el transistor de nuestro circuito que mayor potencia va a disipar
es el transistor $T_{2}$, correspondiente al par dárlington, el cual
por diseño va a tener que soportar una corriente cercana de aproximadamente
1.5 A. Para hallar la potencia máxima que este transistor disipa,
primero debemos saber en qué condiciones va a hacerlo. Para ello,
sabiendo que:
\begin{center}
$\begin{cases}
P_{D_{T2}}=[V_{i}-V_{0}(i_{0})]\cdot i_{o}\, & siendo\,la\,potencia\,que\,disipa\,el\,transistor\,T2\,(1)\\
i_{0}R_{4}=(V_{0}+i_{0}R_{4})\cdot\frac{R_{3}}{R_{3}+R_{4}}+0.7 & ecuaci\acute{o}n\,de\,FoldBack\,(2)
\end{cases}$
\par\end{center}

Debemos de despejar $V_{0}$ de la ecuación de foldback, insertarla
en la ecuación de la potencia del transistor T2 y derivarla respecto
a $i_{0}$ para hallar su máximo.

Haciendo las cuentas pertinentes, notamos que la mayor disipación
de potencia en el transistor $T_{2}$ ocurre cuando se pone la fuente
en corto, es decir cuando $i_{0}=1.5\,A$. Sobredimencionando $V_{i}=22\,V$,
reemplazando los valores en la ecuación (1) obtenemos una potencia
$P_{T_{2}}=22.88\,W$.
\subsubsection{Elección del disipador}

Una vez conocida la potencia máxima disipada, podemos aplicar la ley
de propagación térmica vista previamente para así calcular cuál es
la resistencia térmica que debe tener el transistor. La ecuación pertinente
se puede observar a continuación:
\begin{center}
{\large{}$T_{m\acute{a}x}=T_{a}+P_{max}(\sum R_{\theta i}$)}{\large\par}
\par\end{center}

Donde $T_{max}$ es la temperatura máxima a la cual el componente
puede operar, $T_{A}$ es la temperatura ambiente a la cual está expuesto
el dispositivo, $P_{max}$ es la potencia máxima, y $R_{\theta}$
es la resistencia térmica (de lo/s componentes que correspondan).

Sabemos por la hoja de datos que el transistor TIP 41 posee una resistencia
térmica juntura-ambiente es de un valor $R_{\theta ja}=50\,\frac{\text{ºC}}{W}$
. Sabiendo además que la $T_{max}$ de operación del transistor es
de 150ºC , y suponiendo una $T_{A}=24\text{ºC}$, la temperatura del
dispositivo se expresa según la ecuación de arriba como $T=24+24*50=1224\text{º}$
donde claramente es evidente el uso de un disipador para poder funcionar.

Procedemos entonces a calcular cuánto debe valer la resistencia térmica
de todo el conjunto para que el transistor no se queme. Para ello,
despejamos $R_{\theta}$ de la ecuación de arriba, de manera tal de
que, reemplazando, obtenemos:
\begin{center}
{\large{}$\sum R_{\theta}=\frac{T_{max}-T_{A}}{P_{max}}=5.25\,\frac{\text{ºC}}{W}$}{\large\par}
\par\end{center}

Ahora, con el acoplado del disipador, la $R_{\theta}$ total, está
compuesta por la resistencia juntura-carcasa del transistor; y las
resistencias carcasa-sink, sink-ambiente del disipador. Por la hoja
de datos del transistor, sabemos que la resistencia juntura-carcasa
es de un valor $R_{\theta jc}=1.67\,\frac{\text{ºC}}{W}$ como máximo,
ergo debemos buscar un disipador cuyas $R_{\theta cs}+R_{\theta sa}\leq3.847\,\frac{\text{ºC}}{W}$.

Existen disipadores de varios tamaños, formas y materiales distintos,
por lo que para elegir el correcto, comparamos las resistencias de
dichos disipadores para así poder elegir el óptimo. En particular,
el nuestro debe ser para encapsulados TO220, y con un $R_{\theta}$
relativamente bajo.

Vemos como el modelo del disipador ZD-1, con unas medidas medidas
de base 58 mm, altura 29mm, espesor del núcleo central 3.5mm, distancia
entre aletas de 33mm, genera una superficie total de $525.10\,mm^{2}/mm$
lo que produce un $R_{\theta ca}=3.5\,\text{ºC/W}$, la cual se acopla
perfecto con nuestro diseño.

Habiendo elegido este modelo, ahora la temperatura final del componente
va a ser de $T_{m\acute{a}x}=24+22.88\cdot(3.5+1.67)=130.85ºC$
con lo cual el dispositivo puede operar en todos los rangos posibles
para los que fueron diseñados.

\end{document}
