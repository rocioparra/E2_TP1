\documentclass[e2_tp1_main.tex]{subfiles}

\begin{document}

\section{Power supply rejection ratio (PSRR)}

El factor de rechazo de a fuente de alimentacíon se utiliza para describir que tan inmune es un circuito electrónico a variaciones en la tensión de entrada y se define de la siguiente manera:

\begin{equation}
 PSRR(dB) = 20\log_{10}{(\frac{\Delta V_{fuente}}{\Delta V_{output}})} 
\end{equation}

Acontinuación se analizará dicho factor en la fuente regulada realizada utilizando una fuente de tensión continua de 12V provista con un nivel prominente de ripple($1V_{pp}$) y a su vez variando la carga.
En las siguientes tablas se puede observar los resultados de las simulaciones y las mediciones.


\begin{table}[ht!]
\centering
\begin{tabular}{lllll}
\hline
$R_{load}$ & $V_{in}$ & $Ripple _{in} (V_{pp})$ & $Ripple_{out} (V_{pp})$ & $PSRR(dB)$ \\ \hline
$\infty$ & 12 & 1 & 0.0024 & 52,177 \\
$21\Omega$ & 12 & 1 & 0.3527 & 9,051891
\end{tabular}
\caption{valores de PSRR simulados}
\label{table:psrr}
\end{table}


\begin{table}[ht!]
\centering
\begin{tabular}{lllll}
\hline
$R_{load}$ & $V_{in}$ & $Ripple _{in} (V_{pp})$ & $Ripple_{out} (V_{pp})$ & $PSRR(dB)$ \\ \hline
$\infty$ & 12 & 0.144 & 0.0009 & 44.082 \\
$21\Omega$ & 12 & 1,7 & 0.4 & 12.56
\end{tabular}
\caption{valores de PSRR medidos}
\label{table:psrr2}
\end{table}

Como se puede observar los valores del PSRR de las mediciones presentan diferencias considerables  
que se deben a multiples factores. Uno de estos se debe a que al estar presente esta variación en la tensión de entrada los valores de tensión en los cuales trabajan los diodos y transistores, por lo cual cambian sus puntos de trabajo y polarización de estos ultimos.



 


\end{document}