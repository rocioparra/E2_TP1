\documentclass[e2_tp1_main.tex]{subfiles}

\begin{document}


\section{Rendimiento}

Un par\'ametro fundamental de una fuente de tensi\'on regulada es el rango de tensiones de entrada en el cual la salida se encuentra en regulaci\'on. Idealmente, con cualquier tensi\'on igual o supuerior a la de salida se podr\'ia obtener $V_{O\, REG}$, pero la necesidad de llevar a los transistores del circuito a un punto de trabajo apropiado hace que haya una diferencia de tensi\'on m\'inima entre la entrada y la salida. 

Partiendo desde la salida, la tensión en el nodo común a las resistencias $R_4$ y $R_7$ es igual a $V_o$. Despreciando la tensión que cae en la resistencia $R_4$, puesto que es muy pequeña, asumimos que en el colector del transistor T2 obtenemos la tensión $V_o$.
Sabemos por la hoja de datos del fabricante, que para que el transistor T2 no entre en saturación es de $1 V$como máximo para las condiciones de nuestro circuito, por lo que vamos a asumir que el valor de $V_{ce}=1 V$; ahora, al igual que para T2, la hoja de datos del transistor T1 nos indica que  $V_{ce_{sat}}=600mV$ como valor máximo. 
Ahora, planteando mallas entre los transistores del par darlington, vemos que:
\begin{center}
{\large{}$ V_{ce_{T1}}=V_{ce_{T2}}-V_{BE_{on}}$}{\large\par}
\par\end{center}
Como $V_{ce_{T1}}=0.6V$,$V_{BE_{on}}=0.7V$ despejando obtenemos que $V_{ce_{T2}}=1.4V$ , valor admisible dentro del rango del transistor T2.
Finalmente, $V_{in}$ puede expresarse como:
\begin{center}
{\large{}$ V_{in}=V_{ce_{T2}}+V_{o}$}{\large\par}
\par\end{center}

siendo $V_{ce_{T2}}=1.4 V$.

Cabe destacar que los fabricantes proporcionan una $V_{ce_{sat}}$ mínima, por ende, el valor final puede ser mas bajo. Si se tienen en cuenta los valores mínimos, $V_{ce_{T1}}=0.2V$ y $V_{ce_{T2}}=0.1V$, lo que hace que $ V_{in}=0.8V+V_{o}$.

Por otro lado, hay que verificar tambi\'en que la fuente de corriente se polarice correctamente. Recorriendo la rama que une la entrada, los diodos de la fuente y $T_3$, y yendo desde la salida hasta el colector de $T_3$ (despreciando nuevamente la ca\'ida en $R_4$), se obtiene:
\begin{equation}
	\left\{
	\begin{aligned}
	V_{E_3} &= V_O + V_{BE_1} +  V_{BE_2}\\
	V_{C_3} &= 	V_I - 2V_D +  V_{BE_3}
	\end{aligned}
	\right.
	\Rightarrow V_{EC_3} = V_I - V_O + 2V_D+  V_{BE_1} +  V_{BE_2} -  V_{BE_3}
\end{equation}


Por lo tanto, si no queremos que $T_3$ sature, debe cumplirse que:
\begin{equation}
	  V_I - V_O > V_{EC_3 SAT} + 2V_D +  V_{BE_1} +  V_{BE_2} -  V_{BE_3}
\end{equation}

Si asumimos que todas las $V_{BE}$ y $V_D$ son 0.7V, tomando los 0.3V de la hoja de datos del BC557 como $V_{CE_3 SAT}$, se obtiene que $V_I - V_O > 2.4$V. Sin embargo, emp\'iricamente se verifica que todas estas tensiones son inferiores a 0.7V, y por lo tanto el circuito funciona correctamente para diferencias inferiores.

Tomando como criterio que el circuito est\'a regulando correctamente si la tensi\'on de salida es un 99\% o m\'as de la que se tiene sin carga, se obtuvieron los datos de la tabla \ref{table:rendimiento}.

\begin{table}[htb!]
\centering
\begin{tabular}{|c||c|c|}
\hline
$V_{O\,REG}$ (V) & $V_I - V_O$ simulada (V) & $V_I - V_O$ medida (V) \\ \hline\hline
9                & 1.263                    & 1.37                   \\ \hline
10               & 1.273                    & 1.37                   \\ \hline
11               & 1.281                    & 1.5                    \\ \hline
12               & 1.286                    & 1.18                   \\ \hline
13               & 1.290                    & 1.29                   \\ \hline
14               & 1.293                    & 1.26                   \\ \hline
15               & 1.295                    & 1.13                   \\ \hline
\end{tabular}
\caption{Rendimiento medido y simulado para distintas tensiones de regulaci\'on}
\label{table:rendimiento}
\end{table}

En promedio, pues, se obtuvo que:

\begin{equation}
	\left\{
	\begin{aligned}
	\Delta V_{MIN\,SIM} &= 1.28V \\ 
	\Delta V_{MIN\,MED} &= 1.3V \\ 
	\end{aligned}
	\right.
\end{equation}



\section{Impedancia de salida}

Una caracter\'istica de una fuente de tensi\'on regulada ideal es que posee impedancia de salida cero, lo cual resultar\'ia en regulaci\'on de carga ideal, es decir: la tensi\'on de salida es la misma para cualquier valor de $R_L$. Desde ya, en la pr\'actica esto es imposible de obtener, pero la realimentaci\'on negativa de este tipo de fuentes permite obtener un valor de $R_O$ lo suficientemente peque\~no como para que sea despreciable en la mayor\'ia de los casos. 

Anal\'iticamente, el valor de $R_O$ se obtiene a partir de la impedancia de salida del circuito amplificador b\'asico $R_{OA}$ y la ganancia de lazo $|T|$:

\begin{equation}
	R_O = \frac{R_{OA}}{1+|T|}
	\label{eq:rout}
\end{equation}

En la ecuaci\'on \ref{eq:rout} resulta muy claro por qu\'e la realimentaci\'on negativa mejora esta caracter\'istica del circuito: en lugar de ver directamente la impedancia de salida del circuito amplificador b\'asico, la carga ve a esta impedancia reducida $1+|T|$ veces.

\todo[inline]{un poco de desarrollo matem\'atico sobre esto}

Se obtiene anal\'iticamente entonces que:

\begin{equation}
	\left\{
	\begin{aligned}
	|T| &\simeq ... \\
	R_{OA} &\simeq ...
	\end{aligned}
	\right.
\end{equation}

\todo[inline]{completar modT, Roa}

Remiti\'endonos nuevamente a la ecuaci\'on \ref{eq:rout}, resulta entonces que:

\begin{equation}
	R_O \simeq 4\times 10^{-6}\Omega
\end{equation}

Para simular y medir este valor, se estudi\'o la regulaci\'on de carga del circuito. Si la impedancia de salida fuese cero, deber\'ia cumplirse que tanto para corriente 0 como para cualquier valor de $I_O$, la tensi\'on de salida es la misma. A medida que $R_O$ aumenta, m\'as cae la tensi\'on a medida que aumenta la corriente. Se decidi\'o pues calcular $R_O$ a partir de:

\begin{equation}
	R_O = \frac{V_O|_{I_O=0} - V_O|_{I_O=I_{O\,1}}}{I_{O\,1}}
\end{equation} 

Si bien, en principio, esta ecuaci\'on es v\'alida para cualquier par de valores de $\Delta V_O$ y $\Delta I_O$, se decidi\'o utilizar $I_{O\,1}$=1A, debido a que los valores de $\Delta V_O$ en regulaci\'on son muy peque\~nos, y por lo tanto dif\'iciles de medir con precisi\'on con el instrumental que se ten\'ia disponible, lo cual incurre en un gran error porcentual en la determinaci\'on del mismo, que se propaga proporcionalmente a $R_O$. A pesar de que esta forma se ignora el hecho de que la impedancia de salida cambia con la carga, dado que cambia la polarizaci\'on de todos los diodos y transistores del circuito, se consider\'o que se incurrir\'ia as\'i en menos error que tratando de medir las variaciones de unos pocos mV (o inferiores) que ocurren en valores similares de corriente.

\begin{table}[!htb]
\centering
\begin{tabular}{|c|c|c|}
\hline
$V_{O\,REG}$ (V) & $R_O$ simulada ($\Omega$) & $R_O$ medida ($\Omega$) \\ \hline\hline
9                & $3 \times 10^{-6}$        & $6 \times 10^{-2}$      \\ \hline
15               & $2 \times 10^{-6}$        & $2 \times 10^{-2}$      \\ \hline
\end{tabular}
\caption{Valores simulados y medidos para la impedancia de salida}
\label{table:rout}
\end{table}



\end{document}